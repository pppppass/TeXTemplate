% \iffalse meta-comment
% !TeX encoding = UTF-8
% !TeX program = LuaLaTeX
% Note that pdfLaTeX, XeLaTeX should also work here. When encountering
% difficulties, try changing your LaTeX program. However, TeX generally will
% fail here.
% This file should be named ptmpls.dtx strictly.
%<*internal>
\iffalse
%</internal>
%<*readme>
# ptmpls - pppppass' TeX Templates

This repository ptmpls is a macro set about TeX or LaTeX templates made by
  [pppppass](mailto:lzh2016p@pku.edu.cn).
Held on
  [GitHub](https://github.com/pppppass/ptmpls),
this repository consists of
1. Personal macro package `pdef.sty`,
2. Note class `pnote.cls`,
3. PKU-style paper class `pkupaper.cls`,
4. PKU-style slide class `pkuslide.cls`.

It should be mentioned that the documentation is under construction on the
`doc` branch. It is planned that the macro set will be merged into a single
`ptmpls.dtx` file. Thus, there will be two readme files in the folder at the
same time, one of which is the original one to clarify licenses, and another
is generated from `ptmpls.dtx` and overwrites the original one when
extracting.

To extract the package, simply use the command `lualatex ptmpls.dtx`. For
detailed usages, please refer to the package documentation `ptmpls.pdf`,
together with the preamble and postamble in each file.
%</readme>
%<*internal>
\fi
%</internal>
%<*install>
%% ptmpls.ins - Installation Batchfile

\begingroup
\input docstrip.tex
\askforoverwritefalse
\preamble
ptmpls - pppppass' TeX Templates

Copyright (C) 2017-2018 by pppppass <lzh2016p@pku.edu.cn>

This macro set is held on
  http://www.github.com/pppppass/ptmpls
and released under the LaTeX Project Public License v1.3c or later, see
  http://www.latex-project.org/lppl.txt
and the postamble.

\endpreamble
\postamble
This work may be distributed and/or modified under the
conditions of the LaTeX Project Public License, either version 1.3
of this license or (at your option) any later version.
The latest version of this license is in
  http://www.latex-project.org/lppl.txt,
and version 1.3 or later is part of all distributions of LaTeX
version 2005/12/01 or later.

This work has the LPPL maintenance status "maintained".

The Current Maintainer of this work is pppppass <lzh2016p@pku.edu.cn>.

This work consists of the file
  ptmpls.dtx,
and the derived files
  ptmpls.pdf,
  ptmpls.ins,
  pdef.sty,
  pnote.sty,
while
  Readme.md
and other files not specified belong to the public domain.
\endpostamble
\usedir{tex/latex/ptmpls}
\generate{
  \file{pdef.sty}{\from{ptmpls.dtx}{pdef}}
}
\usedir{tex/latex/ptmpls}
\generate{
  \file{pnote.cls}{\from{ptmpls.dtx}{pnote}}
}
%</install>
%<install>\endbatchfile
%<*internal>
\usedir{source/latex/ptmpls}
\generate{
  \file{ptmpls.ins}{\from{ptmpls.dtx}{install}}
}
\nopreamble\nopostamble
\usedir{doc/latex/ptmpls}
\generate{
  \file{Readme.md}{\from{ptmpls.dtx}{readme}}
}
\endgroup
%</internal>
%<*internal>
\iffalse
%</internal>
%<*pdef>
%% pdef.sty - Personal macro package

%</pdef>
%<*pnote>
%% pnote.cls - Note class

%</pnote>
%<*internal>
\fi
%</internal>
%<*driver>
\documentclass{ltxdoc}
\usepackage{pdef}
\usepackage[numbered]{hypdoc}
\EnableCrossrefs
\CodelineIndex
\RecordChanges
\begin{document}
  \DocInput{ptmpls.dtx}
\end{document}
%</driver>
% \fi
%
% \GetFileInfo{pdef.sty}
%
% \title{\textsf{ptmpls} --- pppppass' Personal \TeX{} Templates}
%
% \author{^^A
%   pppppass^^A
%   \thanks{E-mail: \href{mailto:lzh2016p@pku.edu.cn}{\texttt{lzh2016p@pku.edu.cn}}}^^A
% }
%
%^^A <DATE>
% \date{
%   \begin{tabular}{c}
%     September 3, 2018 \\
%     Version: \filedate, \fileversion
%   \end{tabular}
% }
%
% \maketitle
%
% \changes{v1.2.0}{2018/08/06}{
%   Merge files into \texttt{ptmpls.dtx}, and strip out unnecessary commands
%   according to KISS.
% }
%
% \changes{v1.1.1}{2018/06/19}{
%   Update bibliography sorting and footnotes in \texttt{pkupaper.cls}
% }
% \changes{v1.1.0}{2018/05/16}{
%   Update \texttt{nochinese} option in \texttt{pkupaper.cls} and reformat
%   font sizes.
% }
%
% \changes{v1.0.4}{2018/04/30}{
%   Hot fix for \TeX{} Live 2017.
% }
% \changes{v1.0.3}{2018/04/17}{
%   Introduce some symbols.
% }
% \changes{v1.0.2}{2018/03/22}{
%   Fix \texttt{quoteslide} environment and reformat templates.
% }
% \changes{v1.0.1}{2018/01/20}{
%   Update version header.
% }
% \changes{v1.0.0}{2018/10/25}{
%   First public release.
% }
%
% \tableofcontents
%
% \section{Introduction}
%
% This repository \textsf{ptmpls} is a macro set about TeX / LaTeX templates
% made by
%   \href{mailto:lzh2016p@pku.edu.cn}{pppppass}.
% Held on
%   \href{https://github.com/pppppass/ptmpls}{GitHub},
% this repository consists of
% \begin{enumerate}
% \item Personal macro package \verb"pdef.sty";
% \item Note class \verb"pnote.cls";
% \item PKU-style paper class \verb"pkupaper.cls";
% \item PKU-style slide class \verb"pkuslide.cls".
% \end{enumerate}
%
% This repository is a derived work from some packages under LPPL (LaTeX
% Project Public License). To be exact, this repository contains codes derived
% from
%   \href{https://ctan.org/pkg/yhmath}{\textsf{yhmath}}
% and
%   \href{https://ctan.org/pkg/cbgreek-complete}{\textsf{cbgreek}}.
% See comments on source codes and the preamble and postamble part of extracted
% files for details.
%
% \section{Usage}
%
% There is simple one main file for this repository: \verb"ptmpls.dtx". To
% extract the packages and classes, one needs to \LaTeX{} it by
%\begin{verbatim}
%lualatex ptmpls.dtx
%\end{verbatim}
% or say running LuaLaTeX through this file. As described in this file itself,
% pdfLaTeX
% or XeLaTeX may also work, but \TeX{} programs are likely to fail. \emph{Note
% that files to be extracted, like} \verb"pdef.sty" \emph{will be overwritten!}
%
% It is suggested to download only \verb"ptmpls.dtx" and extract the files on
% one's own. However, a packed version of extracted files is also provided.
% Please refer to
%   \href{https://github.com/pppppass/textmpls/releases}{the release page}
% for details.
%
% To compile the documentation, the simple commmand
%\begin{verbatim}
%lualatex ptmpls.dtx
%\end{verbatim}
% is sufficient. However, if one wants to build the index, another execution of
%\begin{verbatim}
%makeindex -s gind.ist ptmpls.idx
%makeindex -s gglo.ist -o ptmpls.gls ptmpls.glo
%\end{verbatim}
% in required. One may also use
%\begin{verbatim}
%make
%\end{verbatim}
% or
%\begin{verbatim}
%make all
%\end{verbatim}
% to automatically extract and compile all the things.
%
% To use the package for daily work, one may directly place the
% \verb"ptmpls.dtx" into the
% working folder and and extract everything. And then macros like
%\begin{verbatim}
%\usepackage{pdef}
%\end{verbatim}
% or
%\begin{verbatim}
%\documentclass{pnote}
%\end{verbatim}
% should then work smoothly. Options may be added according to the description
% below. One may also download the packed version and unzip the
% contents. For Git users, it is also possible to clone the repository from
% GitHub or create submodules to keep up-to-date, by
%\begin{verbatim}
%git clone git@github.com:pppppass/ptmpls
%\end{verbatim}
% or
%\begin{verbatim}
%git submodule init git@github.com:ppppass/ptmpls
%\end{verbatim}
% If one finds having copies of the same repository disturbing, it is possible
% to use symbolic links to link necessary files and get around license issues.
% A submodule for a repository composed by several separate \LaTeX{} articles
% exemplifies this situation: linking merely \verb"pdef.sty" and
% \verb"pnote.sty" to each folder is sufficient for \LaTeX ing. Additionally,
% \verb"mktexlsr" can also be used, but it is not fully tested. Hopefully, it
% will install the macro set into \verb"tex/latex/ptmpls" and some other
% similar directories.
%
% \section{Personal macro package \texttt{pdef.sty}}
%
% Guard.
%    \begin{macrocode}
%<*pdef>
%    \end{macrocode}
%
%^^A <DATE><VER>
% Provide the package itself.
%    \begin{macrocode}
\NeedsTeXFormat{LaTeX2e}
\ProvidesPackage{pdef}[2018/09/03 v1.2.1 Personal macro package]
%    \end{macrocode}
%
% Decalre macros.
%    \begin{macrocode}
\def\pdef@opt@dec#1{
  \DeclareOption{#1}{
      \expandafter\def\csname pdef@opt@name@#1\endcsname{\relax}
  }
  \expandafter\def\csname ifpdef@opt@#1\endcsname{
      \expandafter\ifcsname pdef@opt@name@#1\endcsname
  }
}

\pdef@opt@dec{yhaccent}
\pdef@opt@dec{cmrgreekup}
\pdef@opt@dec{safebm}
\pdef@opt@dec{nosidefault}
\pdef@opt@dec{nolstdefault}
\pdef@opt@dec{nomacro}
\pdef@opt@dec{paper}

\ProcessOptions\relax

%    \end{macrocode}
%
% Require basic packages.
%    \begin{macrocode}
\RequirePackage{amsmath}
\RequirePackage{amsfonts}
\RequirePackage{amssymb}
\RequirePackage{amsxtra}
\RequirePackage{mathtools}

%    \end{macrocode}
%
% Set matrix column capacity.
%    \begin{macrocode}
\setcounter{MaxMatrixCols}{30}

%    \end{macrocode}
%
% \DescribeMacro{yhaccent}
% \DescribeMacro{\widetilde}
% \DescribeMacro{\widehat}
% \DescribeMacro{\widetriangle}
% \DescribeMacro{\wideparen}
% Provide wide mathematical accents from \textsf{yhmath} package. This snippet
% is directly modified from \textsf{yhmath.sty}, as part of a derived work.
%    \begin{macrocode}
\ifpdef@opt@yhaccent
  \DeclareSymbolFont{largesymbolsyhmath}{OMX}{yhex}{m}{n}
  \DeclareMathAccent{\widetilde}{\mathord}{largesymbolsyhmath}{"65}
  \DeclareMathAccent{\widehat}{\mathord}{largesymbolsyhmath}{"62}
  \DeclareMathAccent{\widetriangle}{\mathord}{largesymbolsyhmath}{"E6}
  \DeclareMathAccent{\wideparen}{\mathord}{largesymbolsyhmath}{"F3}
\fi

%    \end{macrocode}
%
% \DescribeMacro{cmrgreekup}
% \DescribeMacro{\*GREEKCHAR*up}
% Introduce upright Greek letters from CB Greek fonts as from \textsf{cbgreek}
% package. This snippet is modifed from \verb"cbgreek.sty" as part of a derived
% work. Refer to documentation of the package for visual effects. It seems that
% this font suits better with Computer Modern family, while Adobe Symbol from
% package \textsf{upgreek} (probably with option \verb"Symbolsmallscale") looks
% better with Time New Roman family. The new macros are named \verb"\<greek>up"
% here instead of \verb"\up<greek>" in \textsf{upgreek} to avoid conflicts.
%    \begin{macrocode}
\ifpdef@opt@cmrgreekup
  \RequirePackage[LGR,T1]{fontenc}
  \DeclareSymbolFont{greekupcbgreek}{LGR}{cmr}{m}{n}
  \SetSymbolFont{greekupcbgreek}{bold}{LGR}{cmr}{m}{n}
  \DeclareMathSymbol{\alphaup}{\mathord}{greekupcbgreek}{97}
  \DeclareMathSymbol{\betaup}{\mathord}{greekupcbgreek}{98}
  \DeclareMathSymbol{\gammaup}{\mathord}{greekupcbgreek}{103}
  \DeclareMathSymbol{\deltaup}{\mathord}{greekupcbgreek}{100}
  \DeclareMathSymbol{\epsilonup}{\mathord}{greekupcbgreek}{101}
  \DeclareMathSymbol{\zetaup}{\mathord}{greekupcbgreek}{122}
  \DeclareMathSymbol{\etaup}{\mathord}{greekupcbgreek}{104}
  \DeclareMathSymbol{\thetaup}{\mathord}{greekupcbgreek}{106}
  \DeclareMathSymbol{\iotaup}{\mathord}{greekupcbgreek}{105}
  \DeclareMathSymbol{\kappaup}{\mathord}{greekupcbgreek}{107}
  \DeclareMathSymbol{\lambdaup}{\mathord}{greekupcbgreek}{108}
  \DeclareMathSymbol{\muup}{\mathord}{greekupcbgreek}{109}
  \DeclareMathSymbol{\nuup}{\mathord}{greekupcbgreek}{110}
  \DeclareMathSymbol{\xiup}{\mathord}{greekupcbgreek}{120}
  \DeclareMathSymbol{\piup}{\mathord}{greekupcbgreek}{112}
  \DeclareMathSymbol{\rhoup}{\mathord}{greekupcbgreek}{114}
  \DeclareMathSymbol{\sigmaup}{\mathord}{greekupcbgreek}{115}
  \DeclareMathSymbol{\tauup}{\mathord}{greekupcbgreek}{116}
  \DeclareMathSymbol{\upsilonup}{\mathord}{greekupcbgreek}{117}
  \DeclareMathSymbol{\phiup}{\mathord}{greekupcbgreek}{102}
  \DeclareMathSymbol{\chiup}{\mathord}{greekupcbgreek}{113}
  \DeclareMathSymbol{\psiup}{\mathord}{greekupcbgreek}{121}
  \DeclareMathSymbol{\omegaup}{\mathord}{greekupcbgreek}{119}
\fi

%    \end{macrocode}
%
% \DescribeMacro{safebm}
% Set \verb"\hmmax" and \verb"\bmmax" to reduce the number of symbol fonts
% used, while paying some efficiency in turn. If this option is invoked,
% remember to postpone this package until all packages about symbols and fonts
% have been loaded.
%    \begin{macrocode}
\ifpdef@opt@safebm
  \newcommand{\hmmax}{0}
  \newcommand{\bmmax}{0}
  \RequirePackage{bm}
\fi

%    \end{macrocode}
%
% \DescribeMacro{nosidefault}
% Set up defaults for \textsf{siunitx} package.
%    \begin{macrocode}
\ifpdef@opt@nosidefault\else
  \ifdefined\SI
    \let\Si\si
    \let\si\relax
    \sisetup{
      input-symbols = \pi\spi\dots\cdots,
      binary-units = true,
    }
  \fi
\fi

%    \end{macrocode}
%
% Set up defaults for \textsf{listings} package.
% \DescribeMacro{nolstdefault}
%    \begin{macrocode}
\ifpdef@opt@nolstdefault\else
  \ifdefined\lstset
    \lstset{%
      basicstyle=\ttfamily,%
      keywordstyle=\ttfamily\color{blue!75!black},%
      stringstyle=\ttfamily\color{red!75!black},%
      flexiblecolumns%
    }
  \fi
\fi

%    \end{macrocode}
%
% \DescribeMacro{nomacro}
% \DescribeMacro{\rbr}
% \DescribeMacro{\sbr}
% \DescribeMacro{\cbr}
% \DescribeMacro{\pbr}
% \DescribeMacro{\fbr}
% \DescribeMacro{\gbr}
% Define shorthands for paired brackets. The option \verb"nomacro" is used to forbid
% such definitions.
%    \begin{macrocode}
\ifpdef@opt@nomacro\else
  \newcommand{\rbr}[1]{\left( #1 \right)}
  \newcommand{\sbr}[1]{\left[ #1 \right]}
  \newcommand{\cbr}[1]{\left\{ #1 \right\}}
  \newcommand{\pbr}[1]{\left< #1 \right>}
  \newcommand{\fbr}[1]{\left\lfloor #1 \right\rfloor}
  \newcommand{\gbr}[1]{\left\lceil #1 \right\rceil}

%    \end{macrocode}
%
% \DescribeMacro{\abs}
% \DescribeMacro{\norm}
% Define shorthands for norm-like brackets.
%    \begin{macrocode}
  \newcommand{\abs}[1]{\left| #1 \right|}
  \newcommand{\norm}[1]{\left\| #1 \right\|}

%    \end{macrocode}
%
% \DescribeMacro{\mvert}
% \DescribeMacro{\mnorm}
% Define shorthands for middle delimiters.
%    \begin{macrocode}
  \newcommand{\mvert}{\middle\vert}
  \newcommand{\mnorm}{\middle\|}

%    \end{macrocode}
%
% \DescribeMacro{\rsbr}
% \DescribeMacro{\srbr}
% \DescribeMacro{\siobr}
% \DescribeMacro{\soibr}
% \DescribeMacro{\soobr}
% Define brackets for intervals.
%    \begin{macrocode}
  \newcommand{\rsbr}[1]{\left( #1 \right]}
  \newcommand{\srbr}[1]{\left[ #1 \right)}
  \newcommand{\siobr}[1]{\left[ #1 \right[}
  \newcommand{\soibr}[1]{\left] #1 \right]}
  \newcommand{\soobr}[1]{\left] #1 \right[}

%    \end{macrocode}
%
% \changes{v1.2.1}{2018/08/23}{
%   Reintroduce matrix brackets.
% }
% \DescribeMacro{\mnbr}
% \DescribeMacro{\mrbr}
% \DescribeMacro{\msbr}
% \DescribeMacro{\mcbr}
% \DescribeMacro{\mabr}
% \DescribeMacro{\mmbr}
% \DescribeMacro{\mpbr}
% Define brackets for matrices.
%    \begin{macrocode}
\newcommand{\mnbr}[1]{{\begin{matrix} #1 \end{matrix}}}
\newcommand{\mrbr}[1]{{\begin{pmatrix} #1 \end{pmatrix}}}
\newcommand{\msbr}[1]{{\begin{bmatrix} #1 \end{bmatrix}}}
\newcommand{\mcbr}[1]{{\begin{Bmatrix} #1 \end{Bmatrix}}}
\newcommand{\mabr}[1]{{\begin{vmatrix} #1 \end{vmatrix}}}
\newcommand{\mmbr}[1]{{\begin{Vmatrix} #1 \end{Vmatrix}}}
\newcommand{\mpbr}[1]{{\left<\begin{matrix} #1 \end{matrix}\right>}}

%    \end{macrocode}
%
% \DescribeMacro{\bra}
% \DescribeMacro{\ket}
% \DescribeMacro{\braket}
% \DescribeMacro{\bramidket}
% Define shorthands for brackets in physics.
%    \begin{macrocode}
  \newcommand{\bra}[1]{\left< #1 \right\vert}
  \newcommand{\ket}[1]{\left\vert #1 \right>}
  \newcommand{\braket}[2]{\left< #1 \middle\vert #2 \right>}
  \newcommand{\bramidket}[3]{\left< #1 \middle\vert #2 \middle\vert #3 \right>}

%    \end{macrocode}
%
% \DescribeMacro{\bfrac}
% \DescribeMacro{\nvbr}
% Define shorthands for miscellaneous brackets.
%    \begin{macrocode}
  \newcommand{\bfrac}[2]{\left. #1 \middle/ #2 \right.}
  \newcommand{\nvbr}[1]{\left. #1 \right\rvert}

%    \end{macrocode}
%
% \DescribeMacro{\narr}
% \DescribeMacro{\sarr}
% Define shorthands for array environments.
%    \begin{macrocode}
  \newcommand{\narr}[2]{{\begin{array}{#1} #2 \end{array}}}
  \newcommand{\sarr}[2]{{\begin{subarray}{#1} #2 \end{subarray}}}

%    \end{macrocode}
%
% \DescribeMacro{\sd}
% \DescribeMacro{\pd}
% \DescribeMacro{\ud}
% Define shorthands for differential operators.
%    \begin{macrocode}
  \newcommand{\sd}[1][]{\mathop{\mathrm{d}^{#1} \!}}
  \newcommand{\pd}[1][]{\mathop{\partial^{#1} \!}}
  \newcommand{\ud}[1][]{\mathop{\mathrm{D}^{#1} \!}}

%    \end{macrocode}
%
% \DescribeMacro{\se}
% \DescribeMacro{\si}
% \DescribeMacro{\sj}
% \DescribeMacro{\sk}
% \DescribeMacro{\spi}
% Define shorthands for special characters for mathematical constants.
%    \begin{macrocode}
  \newcommand{\se}{\mathrm{e}}
  \newcommand{\si}{\mathrm{i}}
  \newcommand{\sj}{\mathrm{j}}
  \newcommand{\sk}{\mathrm{k}}
  \ifdefined\piup
    \newcommand{\spi}{\piup}
  \else
    \ifdefined\uppi
      \newcommand{\spi}{\uppi}
    \else
      \newcommand{\spi}{\spi}
    \fi
  \fi

%    \end{macrocode}
%
% \DescribeMacro{\ptrel}
% \DescribeMacro{\ptbin}
% \DescribeMacro{\ptop}
% Define shorthands for phantoms. This is useful to typeset aligned
% multi-line equations, where the first equality is left out for indentation.
%    \begin{macrocode}
  \newcommand{\ptrel}[1]{\mathrel{\phantom{#1}}}
  \newcommand{\ptbin}[1]{\mathbin{\phantom{#1}}}
  \newcommand{\ptop}[1]{\mathop{\phantom{#1}}}
\fi

%    \end{macrocode}
%
% \DescribeMacro{nopaper}
% Introduce \textsf{ntheorem} and \textsf{enumitem} package which are common
% for papers. Pass \verb"nopaper" option to suppress.
%    \begin{macrocode}
\ifpdef@opt@paper
  \RequirePackage{ntheorem}
  \RequirePackage{enumitem}

%    \end{macrocode}
%
% \DescribeEnv{partlist}
% \DescribeEnv{partpartlist}
% \DescribeEnv{partpartpartlist}
% Define environments for indented lists.
%    \begin{macrocode}
  \newenvironment{partlist}[1][]
    {\begin{enumerate}[
      itemsep=0pt, label=(\arabic*), wide,
      labelindent=\parindent, listparindent=\parindent, #1%
    ]}
    {\end{enumerate}}
  \newenvironment{partpartlist}[1][]
    {\begin{enumerate}[
      itemsep=0pt, label=(\roman*), wide,
      labelindent=\parindent, listparindent=\parindent, #1%
    ]}
    {\end{enumerate}}
  \newenvironment{partpartpartlist}[1][]
    {\begin{enumerate}[
      itemsep=0pt, label=\alph*., wide,
      labelindent=\parindent, listparindent=\parindent, #1%
    ]}
    {\end{enumerate}}

%    \end{macrocode}
%
% Define theorem styles without numbering as extensions of \textsf{theorem}.
% See \textsf{ntheorem} for details.
%    \begin{macrocode}
  \newtheoremstyle{nonameplain}
  {\item[%
    \theorem@headerfont%
    \hskip\labelsep{##2}%
    \theorem@separator%
  ]}
  {\item[%
    \theorem@headerfont%
    \hskip \labelsep{##2}\ (##3)%
    \theorem@separator%
  ]}
  \newtheoremstyle{nonameindent}
  {\item[%
    \theorem@headerfont%
    \hskip\labelsep\hspace{\parindent}##2%
    \theorem@separator%
  ]}
  {\item[%
    \theorem@headerfont%
    \hskip \labelsep\hspace{\parindent}##2\ (##3)%
    \theorem@separator%
  ]}
  \newtheoremstyle{nonumberindent}
	{\item[%
    \theorem@headerfont%
    \hskip\labelsep\hspace{\parindent}##1%
    \theorem@separator%
  ]}
  {\item[%
    \theorem@headerfont%
    \hskip \labelsep\hspace{\parindent}##1\ (##3)%
    \theorem@separator%
  ]}
\fi

%    \end{macrocode}
%
% Guard.
%    \begin{macrocode}
%</pdef>
%    \end{macrocode}
%
% \section{Note class \texttt{pnote.cls}}
%
% Guard.
%    \begin{macrocode}
%<*pnote>
%    \end{macrocode}
%
%^^A <DATE><VER>
% Provide the document class itself.
%    \begin{macrocode}
\NeedsTeXFormat{LaTeX2e}
\ProvidesClass{pnote}[2018/09/03 v1.2.1 Note class]

%    \end{macrocode}
%
% Declare options.
%    \begin{macrocode}
\def\pnote@opt@dec#1{
  \DeclareOption{#1}{
      \expandafter\def\csname pnote@opt@name@#1\endcsname{\relax}
  }
  \expandafter\def\csname ifpnote@opt@#1\endcsname{
      \expandafter\ifcsname pnote@opt@name@#1\endcsname
  }
}

\pnote@opt@dec{english}
\pnote@opt@dec{nochinese}

\ProcessOptions\relax

%    \end{macrocode}
%
% \DescribeMacro{english}
% \DescribeMacro{nochinese}
% Load classes and pass options. By default Chinese typesetting environment
% is used. Use \verb"english" to change such localization, or use
% \verb"nochinese" to suppress the package \textsf{ctex}.
%    \begin{macrocode}
\ifpnote@opt@english
  \PassOptionsToClass{scheme=plain}{ctexart}
\fi

\ifpnote@opt@nochinese
  \LoadClass[UTF8, 12pt]{article}
\else
  \LoadClass[UTF8, zihao=-4]{ctexart}
\fi

%    \end{macrocode}
%
% Load packages.
%    \begin{macrocode}
\RequirePackage{amsmath}
\RequirePackage{geometry}
\RequirePackage{tocbibind}

%    \end{macrocode}
%
% Set page layout.
%    \begin{macrocode}
\geometry{a4paper, top=1in, bottom=1in, left=1.25in, right=1.25in}

%    \end{macrocode}
%
% Set page style.
%    \begin{macrocode}
\pagestyle{plain}

%    \end{macrocode}
%
% Set default linespread.
%    \begin{macrocode}
\linespread{1.2}\selectfont

%    \end{macrocode}
%
% Set line skips.
%    \begin{macrocode}
\setlength\lineskiplimit{3pt}
\setlength\lineskip{3pt}

%    \end{macrocode}
%
% Allow breaking-page equations.
%    \begin{macrocode}
\allowdisplaybreaks

%    \end{macrocode}
%
% Guard.
%    \begin{macrocode}
%</pnote>
%    \end{macrocode}
%
% \StopEventually{^^A
%   \PrintChanges
%   \PrintIndex
% }
%
% \Finale
