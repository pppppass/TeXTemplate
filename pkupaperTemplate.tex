%% PKU-style paper template
%
%% pkupaperTemplate.tex
%
% Written by pppppass (lzh2016p@pku.edu.cn), this file belongs to
% public domain.
%
% Further information can be found in
%   pkupaper.manifest.md
% and
%   Readme.md
% and an introduction to the whole project is included as well.

% !TeX encoding = UTF-8 Unicode
% !TeX program = LuaLaTeX
% !TeX spellcheck = LaTeX

% Author : pppppass
% Description : PKU-style paper template

\documentclass[english]{pkupaper}

\usepackage[paper, tikz, algorithm]{def}

\usepackage{lipsum}

\newcommand{\cuniversity}{Peking University}
\newcommand{\cthesisname}{Thesis Subject}
\newcommand{\titlemark}{Theis Title}

\usetikzlibrary{graphs, graphdrawing}
\usegdlibrary{trees}

\title{\titlemark}
\author{%
	\begin{tabular}{c}
Someone \\
Information
	\end{tabular}%
}

	\begin{document}

	\maketitle

Here is some content. 这个文档类有基本的中文支持.

\lipsum[1]

\begin{table}[htbp]
\caption{A table} \label{Tbl:ATable}
\centering
\begin{tabular}{c|cc}
A & B & C \\
\hline
D & E & F
\end{tabular}
\end{table}

\begin{thmdefinition}
This is a definition. You may refer to Table \ref{Tbl:ATable} and Figure \ref{Fig:BinaryTreeForest}.
\end{thmdefinition}

\begin{figure}[htbp]
\centering
\begin{tikzpicture}[binary tree layout]
\graph
{
	[fresh nodes]
	A --
	{
		C --
		{
			D -- E[second],
			F
		},
		G
	},
	G -- H[second] -- I -- H[second] -- I
};
\end{tikzpicture}
\caption{A forest of binary trees} \label{Fig:BinaryTreeForest}
\end{figure}

\begin{equation}
a_n = \frac{1}{\spi} \intb{-\spi}{\spi}{ f \rbr{x} \cos n x \sd x } \uppi
\end{equation}

\begin{thmquestion}[label]
This is a question
\end{thmquestion}

\begin{thmquestion}
Another question
\end{thmquestion}

\begin{algorithm}
\SetAlgoLined

\KwData{Two linked list of strictly increasing elements with head $A$ and $B$ respectively}
\KwResult{A linked list of strictly increasing elements with head $C$, representing the union of $A$ and $B$}

\SetKwData{Sa}{S1}
\SetKwData{Sb}{S2}
\SetKwData{Ele}{element}
\SetKwProg{Func}{Function}{}{end}
\SetKwFunction{Dout}{dump\_out}
\SetKwFunction{Din}{dump\_in}
\SetKwFunction{Push}{PUSH}
\SetKwFunction{Pop}{POP}
\SetKwFunction{IsEmpty}{Sempty}
\SetKwFunction{Enqueue}{enqueue}
\SetKwFunction{Dequeue}{dequeue}
\SetKwFunction{QueueEmpty}{queue\_empty}

$A \slar b$\;

\Func{\Dout{}}
{
	\KwData{None}
	\KwResult{Dump stack elements from $S_1$ to $S_2$}
	
	\BlankLine
	
	\While{not \IsEmpty{$S_1$}}
	{
		\Pop{$S_1$, \Ele}\;
		\Push{$S_2$, \Ele}\;
	}
}

\BlankLine

\Func{\Din{}}
{
	\KwData{None}
	\KwResult{Dump stack elements from $S_2$ to $S_1$}
	
	\BlankLine
	
	\While{not \IsEmpty{$S_2$}}
	{
		\Pop{$S_2$, \Ele}\;
		\Push{$S_1$, \Ele}\;
	}
}

\BlankLine

\Func{\Enqueue{\Ele}}
{
	\KwData{Some \Ele}
	\KwResult{Push \Ele to the back of the queue}
	
	\BlankLine
	
	\Push{$S_1$, \Ele}\;
}

\SetAlgoRefName{1}

\caption{Implement a queue by two stacks}
\end{algorithm}

	\end{document}
