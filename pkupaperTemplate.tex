%% PKU-style paper template
%
%% pkupaperTemplate.tex
%
% Written by pppppass (lzh2016p@pku.edu.cn), this file belongs to
% public domain.
%
% Further information can be found in
%   README.md
% and an introduction to the whole project is included as well.

% !TeX encoding = UTF-8 Unicode
% !TeX program = LuaLaTeX
% !TeX spellcheck = LaTeX

% Author : pppppass
% Description : PKU-style paper template

\documentclass[english]{pkupaper}

\usepackage[paper, tikz]{psdef}

\usepackage{lipsum}

\newcommand{\cuniversity}{Peking University}
\newcommand{\cthesisname}{Thesis Subject}
\newcommand{\titlemark}{Theis Title}

\counterwithout{thmquestion}{part}
\counterwithout{equation}{part}

\usetikzlibrary{graphs, graphdrawing}
\usegdlibrary{trees}

\title{\titlemark}
\author{%
	\begin{tabular}{c}
Someone \\
Information
	\end{tabular}%
}

	\begin{document}

	\maketitle

Here is some content.

\lipsum[1]

\begin{table}[htbp]
\caption{A table} \label{Tbl:ATable}
\centering
\begin{tabular}{c|cc}
A & B & C \\
\hline
D & E & F
\end{tabular}
\end{table}

\begin{thmdefinition}
This is a definition. You may refer to Table \ref{Tbl:ATable} and Figure \ref{Fig:BinaryTreeForest}.
\end{thmdefinition}

\begin{figure}[htbp]
\centering
\begin{tikzpicture}[binary tree layout]
\graph
{
	[fresh nodes]
	A --
	{
		C --
		{
			D -- E[second],
			F
		},
		G
	},
	G -- H[second] -- I -- H[second] -- I
};
\end{tikzpicture}
\caption{A forest of binary trees} \label{Fig:BinaryTreeForest}
\end{figure}

\begin{equation}
a_n = \frac{1}{\spi} \intb{-\spi}{\spi}{ f \rbr{x} \cos n x \sd x } \uppi
\end{equation}

	\end{document}
